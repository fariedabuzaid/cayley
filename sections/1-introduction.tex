\section{Introduction}
Data structures are one of the most important concepts in nearly all areas of computer science. Important data structures are, e.g., finite memories, counters, and (theoretically) infinite Turing-tapes. But the most fundamental ones are stacks and queues. And although these two data structures look very similar as they have got the same set of operations on them (i.e.\ writing and reading of a letter), they differ from the computability's point of view: if we equip finite automata with both data structures, then the ones with stacks compute exactly the context-free languages (these are the well-known pushdown automata). But if we equip an finite automaton with queues (in literature they are called queue automata, communicating automata, or channel systems) then we obtain a Turing-complete computation model (cf.~\cite{BraZ83,Bol06}). This strong model can be weakened with various extensions, e.g., if the queue is allowed to forget some of its contents (cf.~\cite{AbdJ96,CecFP96,MasS02}) or if letters of low priority can be superseded by letters with higher priority (cf.~\cite{HaaSS14}).

One possible approach to analyze the difference of the behavior of the data structures is to model them as a monoid of transformations. Then, finite memories induce finite monoids, counters induce the integers with addition, stacks induce the polycyclic monoids (cf. \cite{Sak86,kambites2009}), and queues induce the so-called queue monoids which were first introduced in~\cite{HusKZ17}. And while the transformation monoids of the other data structures are very well-understood, we still do not know much about the queue monoid. Further results on the queue monoid (with and without lossiness) can be found in~\cite{KKP18,Koe18}. Here, we only consider the reliable queue monoids. Concretely, we study the Cayley-graph of this monoid.

Cayley-graphs are a natural translation of finitely generated groups and monoids into graph theory and is a fundamental tool to handle these algebraic constructs in combinatorics, topology, and automata theory. Precisely, these are labeled, directed graphs with labels from a fixed generating set $\varGamma$ of the monoid $\calM$. Thereby, the elements from $\calM$ are the graph's nodes and there is an $a$-labeled edge (where $a\in\varGamma$) from $x\in\calM$ to $y\in\calM$ iff $xa=y$ holds in $\calM$. For groups, we already know many results on their Cayley-graphs. For example, the group's Cayley-graph has decidable first-order theory if, and only if, its existential first-order theory is decidable and if, and only if, the group's word problem is decidable \cite{KusL05}. Moreover, a group's Cayley-graph has decidable monadic second-order theory if, and only if, the group is context-free (that is, if the group's word problem is context-free) \cite{MulS85,KusL05}. Besides these results, Kharlampovich et al.\ considered in \cite{KKM14} so-called Cayley-graph automatic groups (these are the groups having an automatic Cayley-graph in the sense of \cite{KN95}) which links to the rich theory of automatic structures.

Unfortunately, there are not that many studies on Cayley-graphs of monoids. In particular, there are monoids with decidable word problem but undecidable existential first-order theory of their Cayley-graph \cite{NarO90,KusL06}. For finite monoids the Cayley-graphs are finite and, hence, the first- and second-order theories are decidable in polynomial space~\cite{Graedel03}. For polycyclic monoids the Cayley-graphs are automatic, complete $|A|$-ary trees (where $A$ is the underlying alphabet) with an additional node every other node is connected with (this is the zero element resp. error state). Therefore, due to \cite{KusL06} the Cayley-graphs monadic second-order theory is decidable (the first-order theory is even in $\classfont{2EXPSPACE}$ by~\cite{KusL11}).

In this paper we want to consider logics on the Cayley-graph of the queue monoid. Concretely, we will see that this graph's first-order theory is decidable by giving an primitive recursive (but non-elementary) algorithm which combines two well-known methods from model theory in a (at least for the authors) new way: the method of Ferrante and Rackoff~\cite{FerR79} and an automata-based approach. This gives an answer on a question raised by Huschenbett, Kuske, and Zetzsche~\cite{HusKZ17}. There, they conjectured the undecidability of its first-order logic implying that the graph is not automatic in the sense of~\cite{KN95}. Moreover, we will prove the undecidability of the monadic second-order theory with the help of a well-known result from Seese~\cite{See91}.