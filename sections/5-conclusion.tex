\section{Conclusion and Open Problems}
We studied the Cayley-graph of the queue monoid and the logics of these graphs. Concretely, we have shown the decidability of the Cayley-graph's first order theory and the undecidability of the monadic second-order theory. This answers a question from Huschenbett et al.\ in \cite{HusKZ17}.

In Table~\ref{tab:conclusion} is a comparison of our results compared to other fundamental data structures.

\begin{table}[h]
	\begin{tabular}{cc|cc}
		Data Structure & Transformation Monoid $\calM$ & $\FOTh{\cayley{\calM}{\varGamma}}$ & $\MSOTh{\cayley{\calM}{\varGamma}}$\\
		\hline
		\hline
		finite monoid & finite monoid & $\classfont{PSPACE}$ \cite{Graedel03} & $\classfont{PSPACE}$ \cite{Graedel03}\\
		counter & $(\bbZ,+)$ & $\classfont{2EXPSPACE}$ \cite{KusL11} & decidable \cite{KusL06}\\
		stack & polycyclic monoid & $\classfont{2EXPSPACE}$ \cite{KusL11} & decidable \cite{KusL06}\\
		queue & queue monoid & primitive recursive & undecidable
	\end{tabular}
	\caption{Comparison of the decidability of logics on Cayley-graphs of fundamental data structures.\label{tab:conclusion}}
\end{table}

There are still some questions open relating to the queue monoid: in this paper we have given an primitive recursive but non-elementary upper bound on the complexity of the first-order theory of the queue monoid's Cayley-graph. So, one may ask for tight upper and lower bounds.
Another open question concern the automaticity of the queue monoid. While it is neither automatic in the sense of Khnoussainov and Nerode~\cite{KN95} nor automatic in the sense of Thurston et al.~\cite{CEHLPT92} due to~\cite{HusKZ17}, we still do not know whether the Cayley-graph of the queue monoid is automatic.
Finally, the decidability of the first-order theory of the (partially) lossy queue monoid's (cf.~\cite{KKP18,Koe18}) Cayley-graph is left open as well and is worth to be studied.