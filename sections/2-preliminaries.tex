\section{Preliminaries}
Let $A$ be an alphabet. We use $\preceq$ to denote the \emph{prefix-relation} and $\sqsubseteq$ for the \emph{suffix-relation} on $A^\ast$. If $u= vw$ we write $v^{-1}u = w$ and
$uw^{-1} = v$. Thereby, $v$ is the \emph{complementary prefix} of $w$ wrt.\ $u$ and $w$ the \emph{complementary suffix} of $v$ wrt.\ $u$. For $u,v\in A^\ast$ let $u \sqcap v$ denote the largest suffix of $u$ that is also a prefix of $v$.

For $m,n,r\in\bbN$ we write $m =_r n$ iff $m=n$ or $m,n>r$. The function $\exp_{r}(n)$ is inductively defined by $\exp_0(n) = n$ and $\exp_{r+1}(n) = 2^{\exp_r(n)}$.

\paragraph*{Logic on Graphs and Words}
Let $A$ be a finite set of labels. An \emph{edge-labeled graph} is a tuple $G=(V^G,(E_a^G)_{a\in A})$ where $V$ is the set of vertices and $E_a^G \subseteq V\times V$ is the set of $a$-labeled edges. 
A \emph{word-structure} over $A$ is a tuple $W = (\{0,\ldots, n-1\}, \leq^W, (P_a^W)_{a\in A})$ where $\leq^W$ is the usual order on $\{0,\ldots,n-1\}$, and $(P_a^W)_{a\in A}$ is a partition of $\{0,\ldots, n-1\}$ (some of the sets $P_a^W$ may be empty). Whenever we use logic to describe properties of a word $w$ then the formula is evaluated on the corresponding word structure $W$. 

Let $\tau = \{R_1,\ldots,R_m, c_1, \ldots, c_n\}$ where $R_i$ is a relation symbol of arity $r_i$ and $c_j$ is a constant symbol.
\emph{First-order formulas} (over the vocabulary $\tau$) are build up
from variables and constant symbols $\{x_i \mid i\in \bbN \}\cup \{c_1,\ldots,c_n\}$, the edge relation symbols $\{R_1,\ldots, R_m\}$, the equality symbol $=$, the Boolean connectives
$\{\lnot,\vee,\wedge, \to \}$,
quantifiers $\{\forall, \exists \}$, and the bracket symbols
$\{(,) \}$. 
We write $G\models \phi$ to denote that the formula $\phi$ is satisfied by the structure $G$.
The \emph{quantifier rank} $\qr(\phi)$ of a formula $\phi$
is the maximal nesting depth of quantifiers within $\phi$. Two structures
$G$ and $H$ are \emph{$r$-equivalent} (denoted $G\equiv_r H$) if they
cannot be distinguished by any formula of quantifier rank $\le r$. 
For a structure $G$ and two tuples $\vec{p}, \vec{q} \in (V^G)^m$ we write $\vec{p} \equiv_r^G \vec{q}$ or say that $\vec{p}$ and $\vec{q}$ are $r$-equivalent in $G$
whenever $G\models \phi(\vec{p}) \Leftrightarrow G\models\phi(\vec{q})$ for all first-order formulas $\phi$ with $m$ free variables and
quantifier rank at most $r$. For all the above notations we adopt the convention that we omit superscripts whenever this should not lead to any confusion. For instance we write 
$\vec{p} \equiv_r \vec{q}$ when the underlying structure $G$ is clear from the context. 

The \emph{$r$-type} of a structure $G$ is the set of all first-order sentences $\phi$ of quantifier rank at most $r$ such that $G\models \phi$. It is well known that there are up to equivalence only
finitely many sentences of quantifier rank at most $r$. Hence the $r$-type of a structure can be characterized by a sentence, which has also quantifier rank $r$. 

\emph{Ehrenfeucht-Fra\"\i{}ss\'e-relations} (resp. \emph{EF-relations}) for a graph $G = (V, (E_a)_{a\in A})$ are a system $(\calE^r_m)_{r,m\in\bbN}$ where  $\calE^r_m$ is an equivalence relation on $V^m$ and
the following is true for all $r,m\in\bbN$ and $\vec{p},\vec{q} \in V^m$:
\begin{itemize}
	\item If $(p_1,\ldots,p_m) \calE^0_m (q_1,\ldots,q_m)$ then the mapping $p_i \mapsto q_i$ is a partial isomorphism, that is $p_i= p_j \Leftrightarrow q_i=q_j$ and
	$(p_i,p_j)\in E_a \Leftrightarrow (q_i,q_j) \in E_a$ for all $1\leq i,j\leq m$ and all $a\in A$.
	\item If $\vec{p} \calE^{r+1}_m \vec{q}$ then for every $p\in V$ there exists a $q\in V$ such that $(\vec{p}, p) \calE^r_{m+1} (\vec{q}, q)$.
\end{itemize}

Ehrenfeucht-Fra\"\i{}ss\'e-relations are useful to identify $r$-equivalent tuples in a graph. This is formalized in the following theorem.
\begin{theorem}[\!\!\cite{Fra54,Ehr61}]
	Let $G$ be a graph, $(\calE^r_m)_{r,m\in\bbN}$ Ehrenfeucht-Fra\"\i{}ss\'e-relations for $G$, and $\vec{p}, \vec{q}$ $m$-tuples of nodes from $G$. If $\vec{p} \calE^r_m \vec{q}$ then 
	$\vec{p} \equiv_r \vec{q}$.
\end{theorem}